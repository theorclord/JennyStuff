\RequirePackage[l2tabu, orthodox]{nag}
\documentclass[12pt]{article}
\usepackage[utf8]{inputenc} 
\usepackage[T1]{fontenc}
\usepackage[english]{babel} 
\usepackage[margin=2.5cm]{geometry}
\geometry{a4paper}
\usepackage{longtable}
\usepackage{subfigure}
\usepackage[normalem]{ulem}
\usepackage{cleveref}
\usepackage{tabularx}

\usepackage{array,arydshln}
%\setlength\dashlinedash{0.2pt}
%\setlength\dashlinegap{4.5pt}
%\setlength\arrayrulewidth{0.2pt}
\newcolumntype{C}[1]{>{\centering\arraybackslash}m{#1}}
\newcolumntype{R}[1]{>{\raggedleft\arraybackslash}m{#1}}

\usepackage{multicol}
\usepackage[table]{xcolor}
\usepackage{todonotes}
\usepackage{menukeys}
\usepackage{listings}
%----------------------------KODE START---------------------------------------
%\usepackage{listings}
%\usepackage{color}
%\usepackage[usenames,dvipsnames,table]{xcolor}
%\definecolor{gray}{rgb}{0.5,0.5,0.5}
%\definecolor{mauve}{rgb}{0.58,0,0.82}
%\lstset{
%  basicstyle=\footnotesize,
%  numbers=left,
%  numberstyle=\tiny\color{gray},
%  stepnumber=1,
%  numbersep=10pt,
%  backgroundcolor=\color{white},
%  showspaces=false,               % show spaces adding particular underscores
%  showstringspaces=false,         % underline spaces within strings
%  showtabs=false,                 % show tabs within strings adding particular underscores
%  frame=single,                   % adds a frame around the code
%  rulecolor=\color{black},        
%  tabsize=4,
%  captionpos=b,                   % sets the caption-position to bottom
%  breaklines=true,                % sets automatic line breaking
%  breakatwhitespace=false,        % sets if automatic breaks should only happen at whitespace
%  title=\lstname,                   % show the filename of files included with \lstinputlisting;
                                  % also try caption instead of title
%  keywordstyle=\color{mauve},          % keyword style
%  commentstyle=\color{Maroon},       % comment style
%  stringstyle=\color{BlueViolet},         % string literal style
%  escapeinside={\%*}{*)},            % if you want to add LaTeX within your code
%  morekeywords={*,...},              % if you want to add more keywords to the set
%  deletekeywords={...}              % if you want to delete keywords from the given language
%}
%----------------------------KODE SLUT----------------------------------------


\setlength\parindent{0pt} % Makes \noindent standard
\usepackage{graphicx} 
\usepackage{sidecap}
\usepackage{caption}
%\usepackage{subcaption}
\usepackage{float} 
\usepackage{wrapfig} % Allows in-line images if needed
\usepackage{hyperref}
\usepackage{amsmath}
\usepackage{amsfonts}
\usepackage{mathtools}
\hypersetup{colorlinks=false,hidelinks, citecolor=black, urlcolor=black}
\usepackage{csquotes}
\usepackage{comment}
\usepackage{mathtools}
\DeclarePairedDelimiter{\ceil}{\lceil}{\rceil}

\usepackage[dot, autosize, outputdir="dotgraphs/"]{dot2texi}
\usepackage{tikz}
\usetikzlibrary{shapes}
\usepackage{url}
\usepackage{booktabs}
\usepackage{multirow}
\usepackage{longtable}
\setcounter{secnumdepth}{4}
\setcounter{tocdepth}{4}
\usepackage[titletoc]{appendix} % Names appendices "Appendix A"
                                % instead of just A in Contents
\usepackage[bottom]{footmisc}
\usepackage{pdfpages}
\usepackage{algorithm}% http://ctan.org/pkg/algorithms
\usepackage{algpseudocode}% http://ctan.org/pkg/algorithmicx


%\usepackage{lmodern}
\usetikzlibrary{arrows,automata}
\usepackage{verbatim}

\linespread{1.2} 
\graphicspath{{./figures/}} 

% fancy drawings
\usepackage{pgf}
\usepackage{epigraph}

% \epigraphsize{\small}% Default
\setlength\epigraphwidth{8cm}
\setlength\epigraphrule{0pt}

\usepackage{etoolbox}

\makeatletter
\patchcmd{\epigraph}{\@epitext{#1}}{\itshape\@epitext{#1}}{}{}
\makeatother

%\usepackage{boxproof}
%\usepackage{nomencl}
\usepackage{natbib}

\newcommand{\fasto}{\textsc{Fasto} }
\newcommand{\mips}{\textsc{Mips} }
\newcommand{\mars}{Mars }
\makeatletter
\def\BState{\State\hskip-\ALG@thistlm}
\makeatother

%-----------------------------------------------------------------------------
% HEADER AND FOOTER STUFF
%-----------------------------------------------------------------------------
\usepackage{fancyhdr}
\usepackage{lastpage} % Making it possible to write ``Page x of y'' in the footer

\pagestyle{fancy}
\fancyhf{}
% Header stuff below
%\lhead{Jenny-Margrethe Vej} 
\chead{}
%\rhead{rwj935} 
% Footer stuff below
\cfoot{Page \thepage \hspace{1pt} of \pageref{LastPage}} % To the left at the bottom

%-----------------------------------------------------------------------------
\begin{document}

\includepdf[pages={-}]{forside.pdf}

\newpage

\section{Description}
You are invited to participate in a research study on how data collected on smartphones can be leveraged for sleep patterns assessment of the owner. By studying this, we hope to create an algorithm for sleep assessment, and provide design implications for a solution for mobilephones to help combat unhealthy behaviour in students sleeping patterns. \\

You will be given a wearable device, that can track your sleep, includes a pedometer and quantifies your physical activity, e.g., by means of number of steps taken. You will need to wear the device all the day and night. We will collect data from the device into your Android OS mobile phone. Your mobile phone will also get an app installed that will record which other mobile applications you are using and how long, including the pressure of your touch on the screen. No content of visited websites or messages, and so on, will be recorded, just the fact that you are using particular application, e.g., a browser or messaging application. \\

At most every 60 seconds we will also collect automatically and unobtrusively on your phone the following information: current time, location in terms of operator network's cell, cell network's signal strength, phone battery level, charging patterns, current running applications in foreground and background, operator network status and its performance, WiFi network status and Bluetooth network status, screen brightness level and orientation. The phone will also log the information about phone calls and send/received SMSes, MMSes, amount of data being send and received on the network interface. The content of phone calls and sent/received SMSes, MMSes will not be collected and the phone numbers will be anonymised. No audio, video recordings will be made by us, only the logging of data on your mobile phone. We will quantify the time you wear the device and log it.\\

This study will run from November 1, 2015 to July 15, 2016. In this time, you are asked to wear the wearable device and use your smartphone like normal. From November 9 to December 6, 2015 and again from March 30 to April 27, 2016 we will meet each week to interview you about the collected data. The interviews will take place at University of Copenhagen, Nørre Campus (preferably), or over Skype video-conference, agreed upon your convenience. Each interview will be about 30 minutes. \\

We require the participant to own an Android OS mobile phone (Android version 4.0.2) and use it like normal. We require that participants is participating in the weekly interviews in the 2 periods mentioned above. In the beginning of this study, we will also ask for information about your civil status, age, gender and living conditions. If you choose not to participate, you are free to do so at any time. 

\section{Time Involvement}
Your participation will take approximately 38 weeks with a special focus on the 2 times 4 weeks in November and April. Your participation in this experiment will take approximately 30 minutes per a week in those weeks where we have weekly interviews. This schedule depends on when you are free and we will work with you to fit within your schedule. 

\section{Risks and Benefits}
The risks and discomfort associated with participation in this study are no greater than those ordinarily encountered in daily life or while carrying a cell phone or a piece of wearable technology like a modern electronic watch. The benefits, which may reasonably be expected to result from this study, are understanding of your wearable device and its smartphone app usage experience and expectations and understanding of own habits. We cannot and do not guarantee or promise that you will receive any benefits from this study.

\section{Payments}
You will not receive a payment for your participation, but you are allowed to keep the wearable device after the research study is completed at July 15, 2016. 

\section{Participant's Rights}
If you have read this form and have decided to participate in this project, please understand your participation is voluntary and you have the right to withdraw consent or discontinue participation at any time without penalty or loss of benefits to which you are otherwise entitled. You have the right to refuse to answer particular questions. Your individual privacy will be maintained in all published and written data resulting from the study. The results of this research study may be presented at scientific or professional meetings or published in scientific journals. You must be at least 18 years old to participate.

\section{Contact Information}
If you have any questions, concerns or complaints about this research, its procedures, risks and benefits, contact the Author, Jenny-Margrethe Vej \texttt{<jvej@di.ku.dk>}, 0045 21952121. If you can not reach the author, please contact Katarzyna Wac at \texttt{<wac@di.ku.dk>}.

\subsection{Independent Contact}
The research have been approved by SCIENCE Faculty Service on behalf of the Data Protection Agency. If you are not satisfied with how this study is being conducted, or if you have any concerns, complaints, or general questions about the research or your rights as a participant, please contact Lisa Ibenfeldt Schultz, \texttt{<liis@science.ku.dk>}, 0045 29611667 to speak to someone independent of the research team.

\section{Signatures}
A signed and dated copy of this consent form is for you to keep.\\
\newline
\newline

\noindent\begin{tabular}{ll}
\makebox[2.5in]{\hrulefill} & \makebox[2.5in]{\hrulefill}\\
Participant & Date\\[8ex]% adds space between the two sets of signatures
\makebox[2.5in]{\hrulefill} & \makebox[2.5in]{\hrulefill}\\
Author & Date\\
\end{tabular}
%SIGNATURE${ _____________________________}$ DATE$ {____________ }$



\end{document}